Il primo esperimento ha lo scopo di introdurre le procedure di comando di un display TFT mediante scheda Arduino DUE. Si scriverà un programma che realizzi un semplice cronometro a tre stati di funzionamento
\begin{itemize}
    \item \textbf{Stato 0:} il circuito attende la pressione del tasto \textit{T} visualizzando il tempo "0.00 s". Durante l'attesa viene visualizzata la scritta "Press to Start"
    \item \textbf{Stato 1:} nel momento in cui il tasto \textit{T} viene premuto, il timer comincia a misurare il tempo, a step di 50 ms. La misura finisce quando il tasto viene rilasciato. Durante la misura appare la scritta “Release to Stop”
    \item \textbf{Stato 2:} quando il tasto T viene rilasciato, il timer si ferma, e visualizza il tempo totale durante cui il tasto è rimasto premuto. Viene inoltre visualizzata la scritta “Press to Reset”. Una ulteriore pressione del tasto \textit{T} resetta il timer e fa tornare il circuito allo stato 0
\end{itemize}
I componenti necessari a questa esperienza sono:
\begin{itemize}
    \item Scheda Arduino DUE
    \item Breadboard e cavi
    \item Display TFT 3.5" 320x480, \textit{HX8357} Adafruit
    \item Interruttore a bottone, FSM2JART, Cod. RS 745-5185
    \item Resistenza $R=10\text{ k}\Omega,0.25\text{ W}$
\end{itemize}
Il circuito è alimentato mediante porta USB del PC, la quale eroga circa $(\sim 5 V)$.
\subsection{Codice}
La libreria \texttt{Adafruit\_HX8357.h}, insieme alle librerie \texttt{Adafruit\_GFX.h} e \texttt{SPI.h} permette alla scheda arduino di comunicare con il Display HX8357 tramite le funzioni dedicate.
\begin{lstlisting}[frame=single, language=Arduino]
#include <SPI.h>
#include "Adafruit_GFX.h"
#include "Adafruit_HX8357.h"

#define TFT_CS 10
#define TFT_DC 9

Adafruit_HX8357 tft = Adafruit_HX8357(TFT_CS, TFT_DC, -1); 
// TFT_RST set to -1 to tie it to Arduino's reset

#define buttonPin 2

int buttonState = LOW;
int lastButtonState = LOW;

unsigned long startTime = 0;

int stato = 0;
int reset = 0;

void setup(){  
  pinMode(buttonPin, INPUT);
  tft.begin();
  tft.setRotation(3);
  tft.fillScreen(HX8357_BLACK);
  tft.setTextColor(HX8357_GREEN);
  tft.setTextSize(8);
  tft.setCursor(100,0);
  tft.println("TIMER");
  tft.setTextColor(HX8357_BLUE);
  tft.setCursor(100, 80);
  tft.print("0.00 s");
  tft.setCursor(20, 180);
  tft.setTextColor(HX8357_RED);
  tft.setTextSize(5);
  tft.print("Press to Start");
}
\end{lstlisting}
\begin{lstlisting}[frame=single, language=Arduino]
void loop() {
  buttonState = digitalRead(buttonPin);
  if(buttonState == HIGH && lastButtonState == LOW && reset == 1){
    startTime = 0;
    reset = 0;
    stato = 0;
    tft.fillRect(0, 60, 480, 320, HX8357_BLACK);
    tft.setTextSize(8);
    tft.setTextColor(HX8357_BLUE);
    tft.setCursor(100, 80);
    tft.print("0.00 s");
    tft.setCursor(20, 180);
    tft.setTextColor(HX8357_RED);
    tft.setTextSize(5);
    tft.print("Press to Start");
  }
\end{lstlisting}
\clearpage
\begin{lstlisting}[frame=single, language=Arduino]
  if (buttonState == HIGH && lastButtonState == LOW && stato == 0){
    lastButtonState = buttonState;
    stato = 1;
    
    if(startTime == 0){
      startTime = millis();
    }
    
    tft.fillRect(0, 60, 480, 320, HX8357_BLACK);
    tft.setCursor(20, 180);
    tft.setTextSize(5);
    tft.setTextColor(HX8357_RED);
    tft.print("Release to Stop");
  }
  if (stato == 1 && buttonState == HIGH){
    unsigned long centiseconds = (millis() - startTime) / 10;
    unsigned long seconds = centiseconds / 100;
    centiseconds %= 100;

    tft.setTextSize(8);
    tft.setTextColor(HX8357_BLUE);
    tft.setCursor(100, 80);
    tft.print(String(seconds));
    tft.print(".");
    if (centiseconds < 10) {
      tft.print("0");
    }
    tft.print(String(centiseconds));
    tft.println(" s");
    delay(50);
    tft.fillRect(0, 60, 480, 80, HX8357_BLACK);
  }
\end{lstlisting}
\clearpage
\begin{lstlisting}[frame=single, language=Arduino]
  if (buttonState == LOW && lastButtonState == HIGH){
    lastButtonState = buttonState;

    tft.fillRect(0, 60, 480, 320, HX8357_BLACK);
    tft.setCursor(20, 180);
    tft.setTextSize(5);
    tft.setTextColor(HX8357_RED);
    tft.print("Press to Reset");

    unsigned long centiseconds = (millis() - startTime) / 10;
    unsigned long seconds = centiseconds / 100;
    centiseconds %= 100;

    tft.setTextSize(8);
    tft.setTextColor(HX8357_BLUE);
    tft.setCursor(100, 80);
    tft.print(String(seconds));
    tft.print(".");
    if (centiseconds < 10) {
      tft.print("0");
    }
    tft.print(String(centiseconds));
    tft.println(" s");
    reset = 1;
    stato = 3;
    startTime = 0;
  }
}
\end{lstlisting}
E' possibile apprezzare il funzionamento del circuito dal video al \href{}{seguente link errato} 