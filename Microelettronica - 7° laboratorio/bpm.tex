Utilizzando l'array acquisito si è scritto un codice per misurare la frequenza cardiaca, viene visualizzato il corrispondente valore sul display.\\\\
Anzitutto viene definita la soglia \textit{treshhold} con la quale si identificano i picchi del grafico. Dovendo contare i picchi e non gli attraversamenti della soglia si è deciso di creare una funzione apposita che una volta superata la soglia scorre l'array fintantochè i valori sono sopra la soglia.\\\\
Contando il numero di picchi e l'intervallo dal primo all' ultimo picco, si ricavano ricavato i bpm e si disegna una linea rappresentativa della soglia.
\begin{lstlisting}[frame=single, language=Arduino]
int findBPM(int array[]){
    int avg = mean(array);
    int threshold = avg - 20;
    int peaks = 0;
    int timeOfPeaks[arraySize/2];
    for(int i = 0; i < arraySize; i++){
        if(array[i] > threshold){
            timeOfPeaks[peaks] = i;
            peaks++;
            i+=2;
            while(array[i] > threshold){
                i++;
            }
        }
    }
    int FirstPeaks = timeOfPeaks[0] * 20;
    int LastPeaks = timeOfPeaks[peaks-1] * 20;
    float bps = peaks / ((LastPeaks - FirstPeaks)/1000.0);
    int bpm = bps * 60;

    tft.setCursor(30, 240);
    tft.print("BPM: ");
    tft.print(bpm);
    dispThreshold = map(threshold, min, max, graphHeight, 0);
    tft.drawLine(0,dispThreshold,dispWidth,dispThreshold,HX8357_RED);

    return bpm;
}
\end{lstlisting}
