\documentclass{article}
\usepackage{graphicx}
\usepackage{geometry}
\usepackage{amsmath}
\usepackage{float}
\usepackage{xcolor}
\usepackage{listings}
\usepackage{caption}
\usepackage{subcaption}
\usepackage{xparse}
\usepackage{hyperref}
\usepackage{fancyhdr}
\pagestyle{fancy}
\include{arduinoLanguage}
\cfoot{}
\lfoot{Università degli Studi di Padova, Laboratorio di Microelettronica, AA 2022-2023}
\rfoot{\thepage}

%\hypersetup{
%    colorlinks=true,
%    linkcolor=blue,
%    filecolor=magenta,      https://it.overleaf.com/project/646f3dfe8c7d27e771d1b515#
%    urlcolor=cyan,
%    pdftitle={Overleaf Example},
%    pdfpagemode=FullScreen,
%    }

\title{Esperienza di laboratorio\\\textbf{Arduino DUE\\Heart Rate Monitor}}
\author{Gruppo A6\\Giacomo Calabria - 2007964\\Daniele Venturini - 1195858}
\date{26 May 2023}

\begin{document}
    \maketitle
    \tableofcontents
    \clearpage
    \section{INTRODUZIONE}
    Lo scopo dell'esperienza di laboratorio è progettare e realizzare un generatore di forme d'onda basato su amplificatori operazionali e quindi studiarne il funzionamento e il dimensionamento.
\subsection*{Strumentazione necessaria:}
\begin{itemize}
    \item Generatore di forma d'onda arbitraria
    \item Oscilloscopio a 2 canali
    \item Alimentatore da banco
    \item 1 connettore BNC a “T”
    \item 2 connettore BNC maschio/banana femmina
    \item 1 connettore BNC femmina-femmina
    \item 1 cavo BNC
    \item Cavo 1 mm
    \item Spellafili
\end{itemize}
\subsection*{Componenti specifici utilizzati:}
\begin{itemize}
    \item 2 amplificatori operazionali a doppia uscita, codice TL082CP
    \item Diodi Zener 4.3V, codice 1N5229BTR
    \item Resistenze con valore da determinare
    \item Condensatori da $100nF,220nF,1\mu F$ in dotazione
\end{itemize}
    
    \section{Stadio analogico}
    Riportiamo in Figura \ref{fig:Circuit} lo schematico del circuito analogico.
\begin{figure}[H]
    \centering
    \includegraphics[width=\linewidth]{Circuit1.png}
    \caption{Schema circuito }
    \label{fig:Circuit}
\end{figure}
Il circuito è alimentato mediante porta USB del PC, la quale eroga circa $(\sim 5 V)$. Ed è composto da un primo stadio (OPAMP 1) "in cui si va a regolare l'offset" e un secondo stadio in cui vi è un doppio filtro passa-banda (OPAMP 2) che taglia le frequenze inferiori a 0.8 Hz e superiori a 3 Hz.
\subsection{Dimensionamento resistenze LED}
Dimensioniamo le resistenze $R_{IR},R_{RED}$ in modo da ottenere sui due LED rosso e infrarosso, correnti pari a $I_{RED}=15\text{ mA},I_{IR}=$3 mA rispettivamente. Usando i pin high current della scheda Arduino Due che riescono riescono ad erogare una corrente massima di 15 mA
\begin{equation}
    \frac{V_{CC}-V_{f,IR}}{I_{IR}}=100\text{ }\Omega\quad\frac{V_{CC}-V_{f,RED}}{I_{RED}}=103.33\text{ }\Omega
\end{equation}
Dove i valori della tensione di Forward lì abbiamo reperiti dal datasheet e valgono $V_{f,IR}=3$ V e $V_{f,RED}=1.75$ V.
\subsection{Calcolo funzioni di trasferimento}
Calcoliamo ora la funzione di trasferimento del
\begin{itemize}
    \item Filtro passa-alto costituito da $C_1,R_1,R_4$
    \item Filtro passa-basso costituito da $C_2,R_2,R_3$
    \item Filtro passa-banda complessiva
\end{itemize}
\subsection{Dimensionamento resistenze filtro}
Dimensioniamo le resistenze $R_1, R_2, R_3, R_4$ in modo da ottenere:
\begin{itemize}
    \item Il filtro passa-alto costituito da $C_1, R_1, R_4$ abbia frequenza di taglio pari a 0.8 Hz
    \item La tensione DC a riposo ai morsetti $V_+$ sia pari a 0.15 V
    \item Il filtro passa-basso costituito da $R_2, R_3, C_2$ abbia frequenza di taglio pari a 3 Hz
    \item Il guadagno in DC del filtro passa-basso costituito da $R_2, R_3, C_2$ sia pari a 11   
\end{itemize}
Sceglieremo poi il valore commerciale più vicino per le resistenze $R_1, R_2, R_3, R_4$
\subsection{Tracciamento risposta in frequenza filtro}
Con i valori scelti al punto \ref{}, tracciamo la risposta in frequenza di:
\begin{itemize}
    \item Filtro passa-alto costituito da $C_1,R_1,R_4$
    \item Filtro passa-basso costituito da $C_2,R_2,R_3$
    \item Filtro passa-banda complessiva
\end{itemize}
evidenziando le frequenze di taglio ottenute utilizzando i valori commerciali delle resistenze, necessariamente diverse da 0.8 Hz e 3 Hz
    \clearpage
    
    \section{Stadio digitale Arduino}
    Dopo aver montato il circuito in Figura \ref{fig:Circuit}, connettiamo il segnale $V_0$ all'ingresso analogico A0 della scheda Arduino Due. Utilizziamo la scheda custom contenente fotodiodo VIS/IR e diodo LED IR mostrata nella Figura \ref{fig:HRS}, utilizzando il datasheet dei componenti per definire la polarità degli stessi.
\begin{figure}[H]
    \centering
    \includegraphics[width=0.5\linewidth]{Sensor.png}
    \caption{Heart Rate Sensor}
    \label{fig:HRS}
\end{figure}
Per l'analisi della frequenza cardiaca utilizziamo solamente il LED infrarosso.
\subsection{Codice acquisizione}
Scrivere un codice che permetta di acquisire il segnale in uscita al cardiofrequenzimetro, graficandolo sul display TFT allo stesso tempo. Si utilizzi per il momento il LED infrarosso per questa analisi
\begin{lstlisting}[frame=single, language=Arduino]
#include <SPI.h>
#include "Adafruit_GFX.h"
#include "Adafruit_HX8357.h"

#define sensorPin A0
#define PinIR 12
#define PinRED 11
#define TFT_CS 10
#define TFT_DC 9

Adafruit_HX8357 tft = Adafruit_HX8357(TFT_CS, TFT_DC,-1); 

const int displayWidth = 480;
const int arraySize = displayWidth;
const int graphHeight = 200;
const int sampleInterval = 20;

int acquire[arraySize];
\end{lstlisting}
\clearpage
\begin{lstlisting}[frame=single, language=Arduino]
void setup(){
    pinMode(PinIR,OUPUT);
    pinMode(PinRED,OUTPUT);
    analogReadResolution(12); // 12 bit ADC -> 4096 values

    tft.begin();
    tft.setRotation(3);
    tft.fillScreen(HX8357_BLACK);
    tft.setTextSize(3);
    tft.setCursor(30, graphHeight);
    tft.setTextColor(HX8357_WHITE);
    tft.println("Heart Rate Monitor");
}

void loop(){
    tft.setCursor(30, 240);
    tft.setTextColor(HX8357_WHITE);
    digitalWrite(PinIR,HIGH);
    tft.print("Inizio misurazione ...");    delay(1000);    
    tft.print("3"); delay(1000);    
    tft.print("2"); delay(1000);    
    tft.print("1"); delay(1000);
    tft.fillRect(0, 240, displayWidth, 80, HX8357_BLACK);
    acq();
    digitalWrite(PinIR,LOW);
    tft.setCursor(30, 240);
    tft.print("Scalamento della misurazione ...");
    plot_array_scaled(acquire);
    tft.fillRect(0, 240, displayWidth, 80, HX8357_BLACK);
}
\end{lstlisting}
\begin{lstlisting}[frame=single, language=Arduino]
void acq(){
    unsigned long currentMillis = 0;
    tft.fillRect(0, 0, displayWidth, graphHeight, HX8357_BLACK);
    for(int i = 0; i < arraySize; i++){
        acquire[i] = analogRead(sensorPin);
        currentMillis = millis();
        tft.drawPixel(i, map(array[i], 0, 4095, 0, graphHeight), HX8357_WHITE);
        while(millis() < currentMillis + sampleInterval);
    }
}
\end{lstlisting}
\subsection{Codice plotting}
Una volta acquisito l’array di valori, interrompere l’acquisizione. Scrivere una funzione che ri-scali l’array in modo di visualizzarlo sula metà superiore dello schermo, insieme ad eventuali altre scritte (v. esempio in figura); riportare sotto il corrispondente codice
\begin{lstlisting}[frame=single, language=Arduino]
void plot_array_scaled(int array[]){
    rescale();
    tft.fillRect(0, 0, displayWidth, graphHeight, HX8357_BLACK);
    for(int i = 0; i < arraySize; i++){
        tft.drawPixel(i, array[i], HX8357_WHITE);
    }
}
void rescale(){
    int max = findmax(acquire);
    int min = findmin(acquire);

    for(int i = 0; i < arraySize; i++){
        acquire[i] = map(acquire[i], min, max, 0, graphHeight);
    }
}

int findmax(int array[]){
    int max = array[0];
    for(int i = 0; i < arraySize; i++){
        if(array[i] > max)  max = array[i];
    }
    return max;
}

int findmin(int array[]){
    int min = array[0];
    for(int i = 0; i < arraySize; i++){
        if(array[i] < min)  min = array[i];
    }
    return min;
}
\end{lstlisting}    
    \section{Misura dei BPM}
    Utilizzando l'array acquisito si è scritto un codice per misurare la frequenza cardiaca, viene visualizzato il corrispondente valore sul display.\\\\
Anzitutto viene definita la soglia \textit{treshhold} con la quale si identificano i picchi del grafico. Dovendo contare i picchi e non gli attraversamenti della soglia si è deciso di creare una funzione apposita che una volta superata la soglia scorre l'array fintantochè i valori sono sopra la soglia.\\\\
Contando il numero di picchi e l'intervallo dal primo all' ultimo picco, si ricavano ricavato i bpm e si disegna una linea rappresentativa della soglia.
\begin{lstlisting}[frame=single, language=Arduino]
int findBPM(int array[]){
    int avg = mean(array);
    int threshold = avg - 20;
    int peaks = 0;
    int timeOfPeaks[arraySize/2];
    for(int i = 0; i < arraySize; i++){
        if(array[i] > threshold){
            timeOfPeaks[peaks] = i;
            peaks++;
            i+=2;
            while(array[i] > threshold){
                i++;
            }
        }
    }
    int FirstPeaks = timeOfPeaks[0] * 20;
    int LastPeaks = timeOfPeaks[peaks-1] * 20;
    float bps = peaks / ((LastPeaks - FirstPeaks)/1000.0);
    int bpm = bps * 60;

    tft.setCursor(30, 240);
    tft.print("BPM: ");
    tft.print(bpm);
    dispThreshold = map(threshold, min, max, graphHeight, 0);
    tft.drawLine(0,dispThreshold,dispWidth,dispThreshold,HX8357_RED);

    return bpm;
}
\end{lstlisting}
    
    \section{Misura saturazione $O_2$}
    Per la misurazione della saturazione di ossigeno del sange è necessario implementare una formula per il calcolo. Essa si basa sulle differenze di tensione rilevate con luce rossa e con luce IR. Si valuta il rapporto $R$ tra l’assorbimento della componente rossa e infrarossa, secondo la formula seguente
\begin{equation}
    R=\frac{\left(\frac{AC}{DC}\right)_{RED}}{\left(\frac{AC}{DC}\right)_{IR}}=\frac{\frac{\left(V_{\text{max},RED}-V_{\text{min},RED}\right)}{V_{\text{min},RED}}}{\frac{\left(V_{\text{max},IR}-V_{\text{min},IR}\right)}{V_{\text{min},IR}}}
\end{equation}
A partire da R, è possibile calcolare il livello di saturazione utilizzando la formula riportata in Figura \ref{fig:GraphO2}
\begin{figure}[H]
    \centering
    \includegraphics[width=0.7\linewidth]{GraphO2.png}
    \caption{$SpO_2$ pulse oximeter}
    \label{fig:GraphO2}
\end{figure}
\begin{lstlisting}[frame=single, language=Arduino]
float findO2(int array[]){
    int samples = 1000;
    int IR[samples];
    int RED[samples];
    digitalWrite(PinIR,HIGH);
    delay(250);
    for(int i = 0; i < samples; i++){
        IR[i] = analogRead(sensorPin);
        delay(1);
    }
    digitalWrite(PinIR,LOW);
    delay(100);
    digitalWrite(PinRED,HIGH);
    delay(250);
    for(int i = 0; i < samples; i++){
        RED[i] = analogRead(sensorPin);
        delay(1);
    }
    digitalWrite(PinRED,LOW);

    float k = 0.00080586;

    float IRmax = findmax(IR) * k;
    float IRmin = findmin(IR) * k;
    float REDmax = findmax(RED) * k;
    float REDmin = findmin(RED) * k; 

    float R = ((REDmax - REDmin)/REDmin) / ((IRmax - IRmin) / IRmin);
    float O2 = 97.94 + 1.15 * R;
    
    tft.setCursor(30, 280);
    tft.print("O2: ");
    tft.print(O2);
    
    return O2;
}
\end{lstlisting}
    \section{Foto del display}
    Si conclude con una foto del display HX8357 Adafruit, scattata al termine dell'esperienza

\begin{figure}[H]
    \centering
    \includegraphics[width=0.7\linewidth]{IMG_20230526_120648.jpg}
    \caption{Foto del Display HX8357}
    \label{fig:IMG_20230526_120648}
\end{figure}
\end{document}
