Lo scopo dell'esperienza di laboratorio è progettare e realizzare un generatore di forme d'onda basato su amplificatori operazionali e quindi studiarne il funzionamento e il dimensionamento.
\subsection*{Strumentazione necessaria:}
\begin{itemize}
    \item Generatore di forma d'onda arbitraria
    \item Oscilloscopio a 2 canali
    \item Alimentatore da banco
    \item 1 connettore BNC a “T”
    \item 2 connettore BNC maschio/banana femmina
    \item 1 connettore BNC femmina-femmina
    \item 1 cavo BNC
    \item Cavo 1 mm
    \item Spellafili
\end{itemize}
\subsection*{Componenti specifici utilizzati:}
\begin{itemize}
    \item 2 amplificatori operazionali a doppia uscita, codice TL082CP
    \item Diodi Zener 4.3V, codice 1N5229BTR
    \item Resistenze con valore da determinare
    \item Condensatori da $100nF,220nF,1\mu F$ in dotazione
\end{itemize}