Lo scopo dell'esperienza di laboratorio è studiare e valutare mediante misure di laboratorio le proprietà degli stadi di potenza degli amplificatori in classe A e B. Verranno poi valutate le distorsioni di crossover di ciascuno stadio, implementando una metodologia per la riduzione della distorsione di crossover mediante l'uso della retroazione.\\\\
Il circuito realizzato in questa esercitazione è un amplificatore adatto per amplificare il segnale generato da una piccola radio o da un lettore MP3.
\subsection*{Strumentazione necessaria:}
\begin{itemize}
    \item Generatore di forma d'onda arbitraria
    \item Oscilloscopio a 2 canali
    \item Alimentatore da banco
    \item 1 connettore BNC a “T”
	\item 2 connettore BNC maschio/banana femmina
	\item 1 connettore BNC femmina-femmina
	\item 1 cavo BNC
	\item Cavo 1 mm
	\item Spellafili
    \item Lettore MP3
\end{itemize}