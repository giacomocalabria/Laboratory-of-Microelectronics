\documentclass{article}
\usepackage{graphicx}
\usepackage{geometry}
\usepackage{amsmath}
\usepackage{float}
\usepackage{xcolor}
\usepackage{listings}
\usepackage{caption}
%\usepackage{subcaption}
\usepackage{xparse}
\usepackage{hyperref}
\usepackage{amssymb}
\usepackage{verbatim}
\usepackage{fancyhdr}
\pagestyle{fancy}
\usepackage{xspace}
\cfoot{}
\lfoot{Università degli Studi di Padova, Laboratorio di Microelettronica, AA 2022-2023}
\rfoot{\thepage}

\hypersetup{
    colorlinks=true,
    linkcolor=blue,
    filecolor=magenta,      
    urlcolor=cyan,
    pdftitle={Overleaf Example},
    pdfpagemode=FullScreen,
    }

\usepackage{listings}
\include{arduinoLanguage}

\title{Esperienza di laboratorio\\\textbf{Esperimenti con LED e scheda Arduino Due}}
\author{Gruppo A6\\Giacomo Calabria - 2007964\\Daniele Venturini - 1195858}
\date{21 April 2023}

\begin{document}
    \maketitle
    \tableofcontents
    \clearpage
    
    \section*{INTRODUZIONE}
    Lo scopo dell'esperienza di laboratorio è progettare e realizzare un generatore di forme d'onda basato su amplificatori operazionali e quindi studiarne il funzionamento e il dimensionamento.
\subsection*{Strumentazione necessaria:}
\begin{itemize}
    \item Generatore di forma d'onda arbitraria
    \item Oscilloscopio a 2 canali
    \item Alimentatore da banco
    \item 1 connettore BNC a “T”
    \item 2 connettore BNC maschio/banana femmina
    \item 1 connettore BNC femmina-femmina
    \item 1 cavo BNC
    \item Cavo 1 mm
    \item Spellafili
\end{itemize}
\subsection*{Componenti specifici utilizzati:}
\begin{itemize}
    \item 2 amplificatori operazionali a doppia uscita, codice TL082CP
    \item Diodi Zener 4.3V, codice 1N5229BTR
    \item Resistenze con valore da determinare
    \item Condensatori da $100nF,220nF,1\mu F$ in dotazione
\end{itemize}
    \clearpage
    
    \section{Primo esperimento}
    In questo primo esperimento si vuole studiare il comportamento di un convertitore di tensione switching - SMPC - di tipologia Buck ad \textbf{anello aperto}. Il circuito è rappresentato in Figura \ref{fig:Circuit1}. Sono stati utilizzati i seguenti componenti:
\begin{itemize}
    \item Circuito RLC:
    \subitem Resistenza di carico $R_L=50\Omega$, ottenuta dal parallelo di due resistenze da $100\Omega,1W$
    \subitem Induttore L, da scegliere tra i disponibili in lab.
    \subitem Condensatore C, da scegliere tra i disponibili in lab.
\end{itemize}
\begin{figure}[H]
    \centering
    \includegraphics[width=0.5\linewidth]{images/Circuit1.png}
    \caption{Schema circuito}
    \label{fig:Circuit1}
\end{figure}
Il circuito è alimentato dalla tensione $Vi=+8V$.
\subsection{Dimensionamento di \textit{L} e \textit{C} - prelab}\label{ch:Cap1}
Consideriamo una frequenza di switching di $f=50kHz$ e una tensione di ingresso $V_i=+8V$. Determiniamo anzitutto il valore minimo dell'induttanza tale da garantire la condizione di \textit{continuos conduction mode} per un rate di duty cycle maggiore del 20\%. Vogliamo che sia quindi garantita
\begin{equation}
    I_{min}=V_O\left(\frac{1}{R}-\frac{1-\delta}{2L_{min}f}\right)=0\implies L_{min}=\frac{(1-\delta)R}{2f}=400\mu H
\end{equation}
Per cui scegliamo dagli induttori disponibili in laboratorio la seguente induttanza: $L=680\mu H$.\\\\
Dalla scelta di questa induttanza ci aspettiamo i seguenti valori per la minima e massima corrente di induzione
\begin{equation}
    I_{Lmax}=V_o\left(\frac{1}{R}+\frac{1-\delta}{2Lf}\right)=50.82mA
\end{equation}
\begin{equation}
    I_{Lmin}=V_o\left(\frac{1}{R}-\frac{1-\delta}{2Lf}\right)=13.18mA
\end{equation}
Determiniamo adesso un valore appropriato per il condensatore $C$ tale da garantire che l'output voltage ripple $\Delta V_c$ sia più piccolo di $100mV$. Partendo dalla relazione
\begin{equation}
    C=\frac{\Delta I_L}{8f\Delta V_c}\implies C=0.5881\mu F
\end{equation}
Ricaviamo abbiamo scelto il seguente valore per la capacità $C=1\mu F$.





\subsection{Assemblaggi e settaggi}
Dopo aver assemblato il circuito, seguendo lo schematico in Figura \ref{fig:Circuit1SPICE}. Abbiamo alimentato il circuito con il canale 2 dell'alimentatore da banco, in modo da avere la tensione di $V_i=+8V$ necessaria al funzionamento del circuito.
\begin{figure}[H]
    \centering
    \includegraphics[width=\linewidth]{images/Circuit1SPICE.png}
    \caption{Schematico SPICE del circuito}
    \label{fig:Circuit1SPICE}
\end{figure}
Tramite il generatore di funzioni abbiamo fornito al terminale \textit{INP} dell' integrato LTC7001 il seguente segnale:
\begin{itemize}
    \item Forma d'onda: quadra
    \item Ampiezza: 5V picco-picco
    \item Tensione di offset: $V_{offset}=2.5V$
    \item Frequenza: $f=50kHz$
    \item Duty cycle: 50\%
\end{itemize}
\clearpage
\subsection{Risultati}
Riportiamo anzitutto tensioni al gate e al source del transistor, misurate tramite l'oscilloscopio e le relative schermate in Figura \ref{fig:GateMeasure1} e \ref{fig:SourceMeasure1}.
\begin{table}[H]
    \centering
    \begin{tabular}{|c|c|}
        \hline
        $Ampl(G)$&$17.3V$\\\hline
        $Ampl(S)$&$8.7V$\\\hline
    \end{tabular}
\end{table}
\begin{figure}[H]
    \centering
    \includegraphics[width=0.6\linewidth]{images/scope_0.png}
    \caption{Segnale al gate del transistor}
    \label{fig:GateMeasure1}
\end{figure}
\begin{figure}[H]
    \centering
    \includegraphics[width=0.6\linewidth]{images/scope_1.png}
    \caption{Segnale al source del transistor}
    \label{fig:SourceMeasure1}
\end{figure}
Misuriamo ora la tensione di uscita ai capi della resistenza di carico $R$ come funzione del tempo, riportiamo in Figura \ref{fig:OutputLoad1} la schermata dell'oscilloscopio
\begin{figure}[H]
    \centering
    \includegraphics[width=0.6\linewidth]{images/scope_6.png}
    \caption{Tensione di uscita}
    \label{fig:OutputLoad1}
\end{figure}
Misuriamo anche il \textit{ripple} della tensione di uscita, sempre ai capi del carico $R$, riportiamo in Figura \ref{fig:OutputRipple1}  la schermata dell'oscilloscopio. Evidenziamo che la misura del voltage ripple è stata fatta impostando l'oscilloscopio in accoppiamento AC, in modo da eliminare le componenti continue del segnale di uscita.
\begin{figure}[H]
    \centering
    \includegraphics[width=0.6\linewidth]{images/scope_2.png}
    \caption{Apprezzamento del voltage ripple sul segnale di uscita}
    \label{fig:OutputRipple1}
\end{figure}
Notiamo dal grafico in Figura \ref{fig:OutputLoad1} che la media della tensione di uscita calcolata dall'oscilloscopio è di $3.4165V$ che si discosta dal valore atteso di $V_0=\delta V_i=4V$. Dal grafico in Figura \ref{fig:OutputRipple1} notiamo che il voltage ripple misurato è $176mV$ che è diverso dal valore di progetto ($100mV$). Entrambi queste anomalie sono dovute dapprima dalla configurazione a circuito aperto del circuito e in secondo luogo dalle incertezze i valori delle componenti. Notiamo inoltre che il segnale presenta notevole disturbo.
\clearpage


\subsubsection{Misura della corrente sull'induttore}
Usando una sonda di corrente il current probe (clamp it on one of the leads of the inductor), misuriamo ora la corrente che scorre nell'induttori durante pochi periodi di switching. Abbiamo riportato in Figura \ref{fig:CurrentInductor1} l'andamento della corrente misurato dall'oscilloscopio.
\begin{figure}[H]
    \centering
    \includegraphics[width=0.8\linewidth]{images/scope_4.png}
    \caption{Corrente nell'induttore}
    \label{fig:CurrentInductor1}
\end{figure}
Riportiamo infine le misure delle correnti effettuate in modo da compararle con i valori teorici calcolati al punto \ref{ch:Cap1}
\begin{table}[H]
    \centering
    \begin{tabular}{|c|c|c|}
        \hline
        &Valore misurato&Valore teorico\\\hline\hline
        Corrente picco-picco&$68.7mA$&$37.64mA$\\\hline
        Corrente massima&$104.8mA$&$50.82mA$\\\hline
        Corrente minima&$36.0mA$&$13.18mA$\\\hline
    \end{tabular}
\end{table}
Notiamo anche qui un scostamento tra i valori teorici e i valori misurati. Per motivi già citati questi sono dovuti a incertezze nelle componenti e alla configurazione a circuito aperto.
\clearpage
\subsubsection{Dipendenza tensione di uscita - duty cycle}
Vogliamo in quest'ultimo punto valutare la dipendenza tra la tensione di uscita e il duty cycle. Abbiamo raccolto in laboratorio i dati presenti in Tabella \ref{tab:TabPlot}
\begin{table}[H]
    \centering
    \begin{tabular}{|c|c|}
        \hline
        Duty cycle (\%)&Output Voltage $V_o(V)$\\\hline\hline
        $20\%$&$1.0149V$\\\hline
        $30\%$&$1.8014V$\\\hline
        $40\%$&$2.6089V$\\\hline
        $50\%$&$3.4165V$\\\hline
        $60\%$&$4.2398V$\\\hline
        $70\%$&$5.0630V$\\\hline
        $80\%$&$5.2021V$\\\hline
    \end{tabular}
    \caption{Dati raccolti facendo variare il duty cycle del generatore}
    \label{tab:TabPlot}
\end{table}
Abbiamo infine plottato i risultati con MATLAB e prodotto il diagramma in Figura \ref{fig:MatlabPlot1}
\begin{figure}[H]
    \centering
    \includegraphics[width=0.7\linewidth]{images/Ris1TableDTCVo.png}
    \caption{Plot MATLAB dei dati raccolti in Tabella \ref{tab:TabPlot}}
    \label{fig:MatlabPlot1}
\end{figure}
Notiamo che l'andamento, a parte un piccolo scostamento nella misura con duty cycle 80\%, è lineare. Come dopotutto ci aspettiamo dalla relazione $V_o=\delta V_i$
    \clearpage
    
    \section{Secondo esperimento}
    In questo esperimento si vuole controllare un LED RGB tramite l'azionamento di un potenziometro. Per questo esperimento sono stati utilizzati i seguenti componenti:
\begin{itemize}
    \item Resistenze $R_R,R_G,R_B$ da determinare, $0.25$ W. 
    \item LED RGB SMD, codice \textit{ASMB-MTB0-0A3A2}, Avago
    \item 3 transisor NPN, codice \textit{P2N2222AG}, ON Semiconductor
    \item Potenziometro lineare $R_{var}:10\text{ k}\Omega$
    \item Scheda Arduino DUE
    \item Interruttore
\end{itemize}
Il circuito riportato in Figura \ref{fig:Circuit2} è alimentato mediante porta USB del PC, la quale eroga circa $(\sim 5 V)$
\begin{figure}[H]
    \centering
    \includegraphics[width=0.4\linewidth]{images/Circuit2.png}
    \caption{Schema circuito}
    \label{fig:Circuit2}
\end{figure}
\noindent Il potenziometro è collegato al pin analogico \textbf{A0}, mentre i tre led sono collegati ai pin digitali \textbf{12,11,10}.
\subsection{Dimensionamento circuito e analisi - prelab}
La massima corrente erogabile da un pin digitale di Arduino DUE è 
\begin{equation*}
    I_{OUT.MAX}=3\text{ mA}
\end{equation*}
La tensione nominale dei tre diodi LED utilizzati, ricavata dal datasheet vale
\begin{equation*}
    V_{ON.R}=2.1\text{ V}\quad V_{ON.G}=3.1\text{ V}\quad V_{ON.B}=3.1\text{ V}
\end{equation*}
Si vuole ora dimensionare le resistenze di base dei transistor in modo che la corrente erogata dai pin digitali sia pari a $2.5\text{ mA}$, si considera per i calcoli che il transistor abbia $V_{BE}=0.7\text{ V}$, quindi con 
\begin{equation}
    R_B=\frac{V_{CC}-V_{BE}}{2.5\text{ mA}}=1040\text{ }\Omega
\end{equation}
si sono quindi scelti i valori delle resistenze di base sulla base dei componenti disponibili in laboratorio
\begin{equation*}
    R_{B.R}=R_{B.G}=R_{B.B}=1\text{ k$\Omega$}
\end{equation*}
Nell'ipotesi che i transistor funzionino in saturazione, si dimensionano le resistenze di collettore dei tre BJT in modo che la corrente sui LED sia pari a $7 \text{ mA}$, quindi con 
\begin{equation}
    R_{C.x}=\frac{V_{CC}-V_{ON.x}}{7\text{ mA}}
\end{equation}
si sono quindi scelti i valori delle resistenze di collettore sulla base dei componenti disponibili in laboratorio
\begin{equation*}
    R_{C.R}=180\text{ $\Omega$}\quad R_{C.G}=33\text{ $\Omega$}\quad R_{C.B}=33\text{ $\Omega$}
\end{equation*}
\subsection{Codice e misure - 1}
Dopo il montaggio del circuito si è scritto un semplice programma che accende i tre LED in DC. Il codice è molto semplice e non richiede nemmeno l'uso del \texttt{void loop()}.
\begin{lstlisting}[frame=single, language=Arduino]
const int RedLedPin = 12;     // Red LED to digital pin 12
const int GreenLedPin = 11;   // Green LED to digital pin 11
const int BlueLedPin = 10;    // Blue LED to digital pin 10

void setup(){
    pinMode(RedLedPin, OUTPUT);     // Set Red LED pin as output
    pinMode(GreenLedPin, OUTPUT);   // Set Green LED pin as output
    pinMode(BlueLedPin, OUTPUT);    // Set Blue LED pin as output
    
    digitalWrite(RedLedPin, HIGH);  // Turn on Red LED
    digitalWrite(GreenLedPin, HIGH);// Turn on Green LED
    digitalWrite(BlueLedPin, HIGH); // Turn on Blue LED
}

void loop(){} // The void loop is empty
\end{lstlisting}
In queste condizioni sono state effettuate alcune misure.\\ 
Dapprima è stata misurata la caduta di tensione base-emettitore dei tre transistor:
\begin{equation*}
    V_{BE.R}=0.742\text{ V}\quad V_{BE.G}=0.761\text{ V}\quad V_{BE.B}=0.761\text{ V}
\end{equation*}
Si è misurata anche la corrente erogata dalle uscite digitali
\begin{equation*}
    I_{pin12}=2.1\text{ mA}\quad I_{pin11}=2.47\text{ mA} \quad I_{pin10}=2.46\text{ mA} 
\end{equation*}
La tensione di funzionamento misurata dei tre LED è
\begin{equation*}
    V_{ON.R}=1.99\text{ V}\quad V_{ON.G}=2.876\text{ V}\quad V_{ON.B}=2.932\text{ V}
\end{equation*}
Notiamo che le misure delle tensioni di conduzione dei vari LED si discostano dai valori tabulati dal datasheet. Infine si misura la corrente che scorre sui tre LED
\begin{equation*}
    I_{R}=6.97\text{ mA}\quad V_{G}=14.63\text{ mA}\quad V_{B}=12.71\text{ mA}
\end{equation*}
Tali valori si discostano da quelli teorici per vari motivi tra cui:
\begin{itemize}
    \item Il dimensionamento delle resistenze è stato fatto sui valori di tensione di conduzione dei vari LED del datasheet e non su quelli misurati.
    \item Le resistenze stesse hanno delle tolleranze e abbiamo preso i valori di resistenza più prossimi a quelli calcolati.
    \item Le tensioni base-emettitore non sono costanti a 0.7 V. Lo stesso vale per la caduta in conduzione $V_{CE}$
\end{itemize}
\subsection{Codice - 2}
L’ingresso analogico \textbf{A0}, a seconda della posizione del potenziometro $R_{var}$, legge valori compresi tra 0 e 1023. Si vuole  definire un algoritmo che – al ruotare del potenziometro – cambi la combinazione di colori sul LED RGB in maniera graduale. La modalità con cui cambiano i colori è espressa dalla Figura \ref{fig:RGBScale}
\begin{figure}[H]
    \centering
    \includegraphics[width=0.8\linewidth]{images/RGBScale.png}
    \caption{Scala RGB del potenziometro}
    \label{fig:RGBScale}
\end{figure}
Abbiamo diviso la scala RGB in tre zone: da 0 a 341, da 342 a 682 e da 683 a 1023. In ciascuna di queste zone abbiamo fatto variare il valore della componente di colore sulla base del grafico in Figura \ref{fig:RGBScale} facendo variare opportunamente il valore del canale del LED da 0 a 255. Si riporta il codice dell'algoritmo implementato
\begin{lstlisting}[frame=single, language=Arduino]
const int RedLedPin = 12;     // Red LED to digital pin 12
const int GreenLedPin = 11;   // Green LED to digital pin 11
const int BlueLedPin = 10;    // Blue LED to digital pin 10

const int PotPin = A0;  // Pot. for RGB control to analog pin A0

const float K = 0.7478; // approx 255/341
// ...
\end{lstlisting}
\clearpage
\begin{lstlisting}[frame=single, language=Arduino]
// ...
int RedLedBright = 0;   // Red LED brightness
int GreenLedBright = 0; // Green LED brightness
int BlueLedBright = 0;  // Blue LED brightness

void setup(){
    pinMode(RedLedPin, OUTPUT);     // Set Red LED pin as output
    pinMode(GreenLedPin, OUTPUT);   // Set Green LED pin as output
    pinMode(BlueLedPin, OUTPUT);    // Set Blue LED pin as output
}

void loop(){
    int PotValue = analogRead(PotPin);  // Read potentiometer value

    if(PotValue == 0){
        RedLedBright = 255;
        GreenLedBright = 0;
        BlueLedBright = 0;
    } else if(PotValue < 341){
        RedLedBright = 255 - PotValue*K;
        GreenLedBright = 0;
        BlueLedBright = PotValue*K;
    } else if(PotValue < 682){
        RedLedBright = 0;
        GreenLedBright = PotValue*K - 255;
        BlueLedBright = 510 - PotValue*K;
    } else if(PotValue < 1023){
        RedLedBright = PotValue*K - 510;
        GreenLedBright = 765 - PotValue*K;
        BlueLedBright = 0;
    } else {
        RedLedBright = 255;
        GreenLedBright = 0;
        BlueLedBright = 0;
    }

    // Set LED brightness
    analogWrite(RedLedPin, RedLedBright); 
    analogWrite(GreenLedPin, GreenLedBright);
    analogWrite(BlueLedPin, BlueLedBright);   
}
\end{lstlisting}
\clearpage
L'aggiunta della lettura avviene aggiungendo la seguente riga nel \texttt{void setup()} che serve per inizializzare la connessione seriale
\begin{lstlisting}[frame=single, language=Arduino]
void setup(){   [...]
    Serial.begin(9600);
}
\end{lstlisting}
E la seguente riga nel \texttt{void loop()} per stampare sul serial monitor l'intensità (da 0 a 255) del segnale sul LED rosso, verde e blu.
\begin{lstlisting}[frame=single, language=Arduino]
Serial.println("RED: %d GREEN: %d BLUE: %d", RedLedBright, GreenLedBright, BlueLedBright); 
\end{lstlisting}
E' possibile apprezzare il funzionamento del circuito dal video al \href{https://mediaspace.unipd.it/media/Esperimento+2/1_y2jgm47p}{seguente link}
\subsection{Codice - 3}
Nell'ultima parte è stato connesso un secondo potenziometro al pin analogico \textbf{A1} e il programma è stato modificato in modo che questo secondo potenziometro permetta di regolare l’intensità luminosa complessiva della luce emessa dal LED RGB.
\begin{lstlisting}[frame=single, language=Arduino]
const int LuxPin = A2;  // Pot. for brightness control to analog pin A0 

int TotBright = 0;

//************* VOID SETUP *************

void loop(){
    int PotValue = analogRead(PotPin);  // Read potentiometer value

    //************* RGB ALGORITHM *************
    
    int LuxValue = analogRead(LuxPin);  // Read potentiometer value
    TotBright = map(LuxValue, 0, 1023, 0, 100);

    RedLedBright *= TotBright /100;
    GreenLedBright *= TotBright /100;
    BlueLedBright *= TotBright /100;

    // Set LED brightness
    analogWrite(RedLedPin, RedLedBright); 
    analogWrite(GreenLedPin, GreenLedBright);
    analogWrite(BlueLedPin, BlueLedBright);  
}
\end{lstlisting}
E' possibile apprezzare il funzionamento del circuito dal video al \href{https://mediaspace.unipd.it/media/Esperimento+2.1/1_ewlxzlyv}{seguente link}
    \clearpage
    
    \section{Terzo esperimento}
    In questo esperimento si vuole costruire e studiare un amplificatore \textit{push-pull} in classe B con retroazione e valutare l'effetto della retroazione sulla distorsione di crossover. Il circuito, riportato in Figura \ref{fig:Circuito3}, è lo stesso utilizzato nell' esperienza precedente, l'unica modifica attuata è quella di introdurre una retroazione tra l'uscita e il morsetto invertente dell'operazionale del secondo stadio. Il circuito viene alimentato dalla tensione duale $\pm V_{CC}=\pm 12V$.
\begin{figure}[H]
    \centering
    \includegraphics[width=\linewidth]{images/Circuit3.png}
    \caption{Schema circuito}
    \label{fig:Circuito3}
\end{figure}
\subsection{Considerazioni sul circuito}
Attraverso l’implementazione della retroazione si ottiene l’effetto di attenuazione della zona di crossover in uscita, dato che la retroazione è introdotta dopo i transistor, quindi tiene già in considerazione la caduta di tensione sulle giunzioni base-emettitore dei BJT.
\begin{itemize}
    \item Se $V_I=0$ entrambi i transistor sono spenti e $V_0=0$.
    \item Se $V_I > 0.5 V$ $Q_1$ si comporta come un inseguitore di emettitore e $V_O = V_I - V_{beQ1}$, mentre $Q_2$ è spento. 
    \item Se $V_I < -0.5V$ $Q_2$ si comporta come un inseguitore di emettitore e $V_O = V_I + V_{beQ2}$, mentre $Q_1$ è spento.
\end{itemize}
\noindent In questo esperimento è stato scelto l'operazione TL082 tenendo in considerazione il valore elevato dello slew rate dell'integrato. Si riporta in Tabella \ref{tab:slewrate} un semplice confronto con lo slew rate degli operazionali LM741 e LM1458 utilizzati negli esperimenti precedenti.
\begin{table}[H]
    \centering
    \begin{tabular}{|c|c|}
        \hline
        Integrato&Slew rate\\\hline
        TL082&$13v/\mu s$\\\hline
        LM741&$0.5v/\mu s$\\\hline
        LM1458&$0.5v/\mu s$\\\hline
    \end{tabular}
    \caption{Slew rate degli integrati}
    \label{tab:slewrate}
\end{table}
Si preferisce utilizzare operazionali con slew rate elevato in quanto ad alte frequenze uno slew rate troppo basso comporta un'accensione/spegnimento continuo dei transistor, che è quello che si vuole evitare.
\subsection{Assemblaggi e settaggi}
Il generatore di forma d'onda è stato impostato con il seguente segnale:
\begin{itemize}
    \item Forma d'onda: sinusoidale
    \item Ampiezza iniziale: $100mV$ picco-picco
    \item Frequenza: $330Hz$ (nota Mi)
\end{itemize}
\subsection{Procedura di valutazione e risultati}
Dopo l'accensione dell'alimentazione, l'oscilloscopio è stato impostato in modo da visualizzare il segnale di ingresso e il segnale di uscita. Il potenziometro che regola il volume è stato regolato in modo che il segnale di uscita raggiunga un'ampiezza di $1V_{pp}$.\\\\
In Figura \ref{fig:scope_12} la relativa forma d'onda dei segnali di ingresso e uscita
\begin{figure}[H]
    \centering
    \includegraphics[width=0.7\linewidth]{images/scope_12.png}
    \caption{Segnali di ingresso e uscita dell'amplificatore in classe B con retroazione}
    \label{fig:scope_12}
\end{figure}
Si è proseguito misurando, attraverso i cursori dell’oscilloscopio, l’effetto della distorsione di crossover. Tuttavia, come si vede in Figura \ref{fig:scope_13} l'effetto di crossover è stato completamente eliminato dalla nuova configurazione del circuito.
\begin{figure}[H]
    \centering
    \includegraphics[width=0.7\linewidth]{images/scope_13.png}
    \caption{Dettaglio segnale di uscita}
    \label{fig:scope_13}
\end{figure}
L’unica differenza introdotta nel circuito è il feedback fornito all’amplificatore operazionale, antecedente il push-pull di uscita. Questo tipo di retroazione consente al segnale di ingresso di operare già tenendo conto della caduta di tensione ai capi del transistor, ottenendo così un valore di uscita che consente di bypassare parte della distorsione del crossover dovuta alla zona morta di attivazione del BJT.
    \clearpage
    
    \section{Quarto esperimento}
    In questo esperimento si vuole studiare un amplificare in classe AB. Il circuito, riportato in Figura \ref{fig:CircuitFac4}, è alimentato dalla tensione duale $\pm V_{CC}=\pm 12V$.
\begin{figure}[H]
    \centering
    \includegraphics[width=0.3\linewidth]{images/CircuitFac4.png}
    \caption{Schema circuito}
    \label{fig:CircuitFac4}
\end{figure}
I due diodi PN (polarizzati dalle resistenze $R_3$ e $R_4$) creano una differenza di potenziale circa uguale a $0.6 V$ tra il segnale di ingresso e la base dei transistor $Q_1$ e $Q_2$. Di conseguenza, basta che il segnale di ingresso superi di poco la tensione nulla perché uno dei due transistor di uscita entri in conduzione. Rispetto allo stadio in classe B, questo schema presenta una distorsione di crossover trascurabile.
\subsubsection{Dimensionamento circuito}
Si è scelto i valori di $R_3$ e $R_4$ in modo tale da da garantire una corrente di $1 mA$ sui diodi $D_1$ e $D_2$ (con tensione di ingresso nulla).\\
Si è scelto quindi
\begin{equation*}
    R_3=R_4=10k\Omega
\end{equation*}
\subsection{Procedura di valutazione e risultati}
Mantenendo la tensione in ingresso $V_{in}=0$, con il multimetro sono state misurate le correnti sulle resistenze $R_3$ e $R_4$
\begin{equation*}
    I_{R_3}=1.14mA\quad I_{R_4}=1.21mA
\end{equation*}
e la caduta di tensione sui due diodi $D_1$ e $D_2$
\begin{equation*}
    V_{D_1}=0.612V\quad V_{D_2}=0.606V
\end{equation*}
\clearpage 
Il generatore di forma d'onda è stato impostato con il seguente segnale:
\begin{itemize}
    \item Forma d'onda: sinusoidale
    \item Ampiezza iniziale: $1V$ picco-picco
    \item Frequenza: $220Hz$ (nota La)
\end{itemize}
Abbiamo impostato l'oscilloscopio per misurare il segnale di ingresso e di uscita. Riportiamo in Figura \ref{fig:scope_15} la schermata dell'oscilloscopio con i due segnali
\begin{figure}[H]
    \centering
    \includegraphics[width=0.7\linewidth]{images/scope_15.png}
    \caption{Segnale in ingresso e in uscita dal circuito}
    \label{fig:scope_15}
\end{figure}
Attraverso i cursori dell'oscilloscopio è stata misurata la distorsione di crossover. Come si vede dal dettaglio della Figura \ref{fig:scope_16} è risultato che non è presente alcuna distorsione. Infatti entrambi i diodi a riposo sono polarizzati in zona diretta e permettono di avere i BJT in zona attiva.
\begin{figure}[H]
    \centering
    \includegraphics[width=0.7\linewidth]{images/scope_16.png}
    \caption{Dettaglio segnale di uscita}
    \label{fig:scope_16}
\end{figure}

\section{Conclusioni}
Commentare brevemente i risultati ottenuti nei tre esperimenti; confrontare i tre stadi amplificatori in termini di prestazioni e efficienza (a livello teorico)
    \clearpage

    \section{Etilometro}
    Si vuole realizzare mediante Arduino un circuito che stimi la concentrazione dell’alcool e in base al valore letto dal sensore \textit{MQ-3} regoli l’accensione delle luci di una LED bar. Sono stati utilizzati i seguenti componenti:
\begin{itemize}
    \item Resistenze $R_1$, $R_2$, $R_0=\dots=R_n$ da determinare, 0.25 W.
    \item LED bar,  codice \textit{DC10SRWA}, Kingbright 
    \item Sensore alcool \textit{MQ-3}
    \item Scheda Arduino DUE
\end{itemize}
Il circuito riportato in Figura \ref{fig:CircuitFacAlcool} è alimentato mediante porta USB del PC, la quale eroga circa $(\sim 5 V)$
\begin{figure}[H]
    \centering
    \includegraphics[width=0.7\linewidth]{images/CircuitFacAlcool.png}
    \caption{Schema circuito}
    \label{fig:CircuitFacAlcool}
\end{figure}
I vari LED della LED bar a 10 segmenti sono collegati ai pin digitali \textbf{12\dots3} mentre il sensore \textit{MQ-3} è collegato al pin analogico \textbf{A0} tramite il partitore di tensione $R_1$ e $R_2$.
\subsection{In laboratorio}
Le resistenze $R_1$ e $R_2$ sono state dimensionate in modo da ottenere:
\begin{itemize}
    \item Tensione in ingresso all’Arduino pari a $3.3\text{ V}$ 
    \item Una resistenza totale di $200\text{ k}\Omega$
\end{itemize}
Quindi sono state scelte
\begin{equation*}
    R_1 = 56 \text{ k} \Omega\quad R_2 = 150 \text{ k} \Omega
\end{equation*}
Successivamente, sono state dimensionate le resistenze $R_0=\dots=R_n$, sapendo che la massima corrente erogata dai pin digitali è $3\text{ mA}$ e la tensione operativa dei singoli LED della LED bar è $2\text{ V}$  [controllare!!]
\begin{equation*}
    R_0=\dots=R_n = 470 \text{ k} \Omega
\end{equation*}
\subsection{Codice}
Infine si è scritto il codice necessario per realizzare l’etilometro. Si vuole che le luci della LED bar si accendano in accordo al valore letto dal sensore. L’obiettivo è di avere 10 luci accese quando la concentrazione è massima e 1 luce accesa quando la concentrazione è minima.
\begin{lstlisting}[frame=single, language=Arduino]
const int alcolPin = A0;
const int matrixPin[10] = {12,11,10,9,8,7,6,5,4,3};

void setup(){
    for(int i = 0; i < 10; i++){
        pinMode(matrixPin[i], OUTPUT);
        digitalWrite(matrixPin[i], LOW);
    }
}

void loop(){
    int alcolValue = analogRead(alcolPin);
    int alcolValue2 = map(alcolValue, 940, 990, 0, 9);
    for(int i = 0; i < 10; i++){
        if(i < alcolValue2){
            digitalWrite(matrixPin[i], HIGH);
        }else{
            digitalWrite(matrixPin[i], LOW);
        }
    }
}
\end{lstlisting}
Abbiamo verificato il corretto funzionamento del programma avvicinando un batuffolo di cotone imbevuto di alcool al sensore.
\end{document}
